\documentclass[a4paper, 13pt]{extreport}
\usepackage{amsmath}
\usepackage{amssymb}
\usepackage{amsthm}
\usepackage{graphicx}
\usepackage{fdsymbol}
\usepackage{enumerate}
\usepackage[left=1cm,right=1cm,top=1cm,bottom=1cm]{geometry}
\usepackage{hyperref}
\usepackage[utf8]{vietnam}  
\usepackage{type1cm}                    % break the font size rule
\usepackage{times}                      % times new roman
\usepackage{fancyhdr}                   % page layout
\usepackage{pdfpages}                   % change pdf dimension
\usepackage{titlesec}                   % modified section, etc
\usepackage{tikz}                       % draw title page border
\usetikzlibrary{calc}                   % draw title page border
\usepackage{array}
\usepackage{tabularx}
\usepackage{booktabs}
\usepackage{ltablex}
\usepackage{diagbox}
\usepackage{slashbox}
\usepackage{xurl}                       % using for large URL

\setlength{\parindent}{1.5cm}
\addtocontents{toc}{\protect\thispagestyle{empty}}
\pdfpagewidth=\paperwidth
\pdfpageheight=\paperheight
\renewcommand{\labelitemii}{\(\smallcircle\)}
\newcommand{\TitleName}{Big Data Analytics Technology}
\newcommand{\student}{Index Huynh}
\newcommand{\Subject}{IE405.E31}
\newcommand{\Teacher}{Lecturer: M.S. Nguyễn Hồ Duy Trí}
\newcommand{\FontB}[3]
    {\normalfont\bfseries\fontsize{#1}{#2}\selectfont #3}
\newcommand{\FontN}[3]
    {\normalfont\fontsize{#1}{#2}\selectfont #3}
\newcommand{\maintitle}
    {\FontB{18pt}{20pt}{\Subject} \\ \FontB{18pt}{20pt}{\TitleName} \\ \FontB{14pt}{20pt}{\student} \\ \FontB{14pt}{20pt}{\Teacher}}

\newcommand{\bottitle}
    {\FontB{13pt}{20pt}{HCMC, \today}}

\newcommand{\bordertitle}{
    \begin{tikzpicture}[remember picture,overlay,inner sep=0,outer sep=0]
        \draw[black,line width=2.5pt] 
            ([xshift=-0.9cm,yshift=-0.9cm]current page.north east) coordinate (A)--
            ([xshift= 0.9cm,yshift=-0.9cm]current page.north west) coordinate (B)--
            ([xshift= 0.9cm,yshift= 0.9cm]current page.south west) coordinate (C)--
            ([xshift=-0.9cm,yshift= 0.9cm]current page.south east) coordinate (D)--cycle;
            
        \draw[black,line width=0.5pt] 
            ([xshift=-1cm,yshift=-1cm]current page.north east) coordinate (A)--
            ([xshift= 1cm,yshift=-1cm]current page.north west) coordinate (B)--
            ([xshift= 1cm,yshift= 1cm]current page.south west) coordinate (C)--
            ([xshift=-1cm,yshift= 1cm]current page.south east) coordinate (D)--cycle;
    \end{tikzpicture}
}
\newcommand{\tab}{\hspace{\parindent}}
%-------------------------------------------------------------------------------
\begin{document}
\begin{titlepage}
\bordertitle
\begin{center}
\vspace{\fill} \maintitle \\ \vspace{\fill} \bottitle
\end{center}
\end{titlepage}
%-------------------------------------------------------------------------------
\newpage
\newgeometry{left=3cm,right=2cm,top=1cm,bottom=2cm}
\tableofcontents
\thispagestyle{empty}
%-------------------------------------------------------------------------------
\newpage
\chapter*{ Preface }
\thispagestyle{empty}
%-------------------------------------------------------------------------------
\vspace{\fill}
\begin{itemize}
\item 
\end{itemize}
\vspace{\fill}
%-------------------------------------------------------------------------------
\newpage
\setcounter{page}{1}
\chapter{ Introduction }
%-------------------------------------------------------------------------------
\section{Introduction and history}
    \textbf{Steganography} is the art of hidden writing, a term derived from
    Greek. It's the art and science of communicating in a way that conceals the
    very existence of the communication itself. The goal is to ensure that an
    outsider, or third party, doesn't realize that a secret message is being
    transmitted.
\section{Fundamental concepts}
    \textbf{Data hiding} is an overarching concept that refers to methods of
    concealing information within another medium. Its purpose is to protect
    information from unauthorized access and to maintain its confidentiality
    and integrity.\\
    
    \textbf{Steganography} and \textbf{watermarking} are two specialized
    sub-branches of data hiding, which differ primarily in their objectives.
\section{Differentiating techniques}
    \textbf{Steganography} aims to conceal the very existence of a message,
    prioritizing un-detectability. The system is considered to have failed if
    the message is discovered.\\
    
    In contrast, \textbf{digital watermarking} seeks to embed an identifying
    mark into data, prioritizing robustness. For the system to be successful,
    the watermark must survive attacks intended to remove or destroy it.\\
    
    \textbf{Cryptography} aims to protect the content of a message. It
    transforms readable text into an unreadable form (ciphertext), making it
    obvious that a message exists, but its meaning remains a secret.
\section{Introduction to techniques and applications}
    The following is a distinction of steganography methods based on the host
    medium:
    \begin{itemize}
        \item \textbf{Text Steganography:} Considered the most difficult due to
            the limited amount of redundant data available for embedding.
        \item \textbf{Image Steganography:} The most popular and widely
            researched method, as images provide a suitable medium for
            concealment.
        \item \textbf{Audio Steganography:} Involves embedding data into bits
            or frequencies that are imperceptible to the human ear.
        \item \textbf{Video Steganography:} Leverages the large data capacity
            of video by combining both image and audio steganography techniques.
        \item \textbf{Network Steganography:} Involves hiding information within
            network packets, often in fields like TCP/IP headers.
    \end{itemize}
\section{Basic Technique: LSB Substitution}
    \textbf{LSB substitution} (Least Significant Bit substitution) is a
    fundamental and widely-used steganography technique. It works by replacing
    the least significant bit (the last bit) of the bytes in a cover medium
    (like an image or audio file) with the bits of the secret message.\\
    
    This method is effective because the least significant bit has the smallest
    impact on the overall value of a byte. For example, in an image, changing
    the LSB of a pixel's color value results in a negligible color shift that is
    typically imperceptible to the humman eye.
    \subsection{How it works}
        Imagine a single pixel's red color value is represented by the byte
        \(11010010\). The LSB is the last \(0\). To embed a secret message bit,
        say a \(1\), you would simply replace the LSB, changing the byte to
        \(11010011\). This minor alteration is difficult for an observer to
        notice without specialized tools, effectively hiding the seret message.
    \subsection{Advantages and Disadvantages}
        It is simple to understand and easy to iplement. But it is not very
        robust. The hidden data can be easily destroyed by common operations
        like image compression or simple steganalysis attacks.
%-------------------------------------------------------------------------------
\end{document}
