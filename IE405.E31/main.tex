\documentclass[a4paper, 13pt]{extreport}
\usepackage{amsmath}
\usepackage{amssymb}
\usepackage{amsthm}
\usepackage{graphicx}
\usepackage{fdsymbol}
\usepackage{enumerate}
\usepackage[left=1cm,right=1cm,top=1cm,bottom=1cm]{geometry}
\usepackage{hyperref}
\usepackage[utf8]{vietnam}  
\usepackage{type1cm}                    % break the font size rule
\usepackage{times}                      % times new roman
\usepackage{fancyhdr}                   % page layout
\usepackage{pdfpages}                   % change pdf dimension
\usepackage{titlesec}                   % modified section, etc
\usepackage{tikz}                       % draw title page border
\usetikzlibrary{calc}                   % draw title page border
\usepackage{array}
\usepackage{tabularx}
\usepackage{booktabs}
\usepackage{ltablex}
\usepackage{diagbox}
\usepackage{slashbox}
\usepackage{xurl}                       % using for large URL

\setlength{\parindent}{1.5cm}
\addtocontents{toc}{\protect\thispagestyle{empty}}
\pdfpagewidth=\paperwidth
\pdfpageheight=\paperheight
\renewcommand{\labelitemii}{\(\smallcircle\)}
\newcommand{\TitleName}{Big Data Analytics Technology}
\newcommand{\student}{Index Huynh}
\newcommand{\Subject}{IE405.E31}
\newcommand{\Teacher}{Lecturer: M.S. Nguyễn Hồ Duy Trí}
\newcommand{\FontB}[3]
    {\normalfont\bfseries\fontsize{#1}{#2}\selectfont #3}
\newcommand{\FontN}[3]
    {\normalfont\fontsize{#1}{#2}\selectfont #3}
\newcommand{\maintitle}
    {\FontB{18pt}{20pt}{\Subject} \\ \FontB{18pt}{20pt}{\TitleName} \\ \FontB{14pt}{20pt}{\student} \\ \FontB{14pt}{20pt}{\Teacher}}

\newcommand{\bottitle}
    {\FontB{13pt}{20pt}{HCMC, \today}}

\newcommand{\bordertitle}{
    \begin{tikzpicture}[remember picture,overlay,inner sep=0,outer sep=0]
        \draw[black,line width=2.5pt] 
            ([xshift=-0.9cm,yshift=-0.9cm]current page.north east) coordinate (A)--
            ([xshift= 0.9cm,yshift=-0.9cm]current page.north west) coordinate (B)--
            ([xshift= 0.9cm,yshift= 0.9cm]current page.south west) coordinate (C)--
            ([xshift=-0.9cm,yshift= 0.9cm]current page.south east) coordinate (D)--cycle;
            
        \draw[black,line width=0.5pt] 
            ([xshift=-1cm,yshift=-1cm]current page.north east) coordinate (A)--
            ([xshift= 1cm,yshift=-1cm]current page.north west) coordinate (B)--
            ([xshift= 1cm,yshift= 1cm]current page.south west) coordinate (C)--
            ([xshift=-1cm,yshift= 1cm]current page.south east) coordinate (D)--cycle;
    \end{tikzpicture}
}
%-------------------------------------------------------------------------------
\begin{document}
\begin{titlepage}
\bordertitle
\begin{center}
\vspace{\fill} \maintitle \\ \vspace{\fill} \bottitle
\end{center}
\end{titlepage}
%-------------------------------------------------------------------------------
\newpage
\newgeometry{left=3cm,right=2cm,top=1cm,bottom=2cm}
\tableofcontents
\thispagestyle{empty}
%-------------------------------------------------------------------------------
% \newpage
% \chapter*{ LỜI MỞ ĐẦU }
% \thispagestyle{empty}
%-------------------------------------------------------------------------------
% \vspace{\fill}

% \vspace{\fill}
%-------------------------------------------------------------------------------
\newpage
\setcounter{page}{1}
\chapter{ Giới thiệu }
%-------------------------------------------------------------------------------
\section{ Mô tả môn học }
\begin{itemize}
    \item \textbf{Kiến thức cần thiết}: toán rời rạc và toán thống kê.
    \item \textbf{Kiến thức được học}: Cung cấp kiến thức nền tảng về dữ liệu lớn, các mô hình và công cụ xử lý dữ liệu lớn.
    \item \textbf{Kiến thức tự học:} kiến thức lĩnh vực (tự trang bị dữ liệu), vd như làm ở công ty nào thì học kiến thức của công ty đó. 
    \item \textbf{Ngôn ngữ thực hành:} Python
\end{itemize}
\section{ Mục tiêu }
\begin{itemize}
    \item \textbf{Hiểu và trình bày được các mô hình, khái niệm cơ bản.}
    \item Sử dụng được các nền tảng lưu trữ dữ liệu lớn. (tổ chức và lưu trữ)
    \item Áp dụng giải thuật khai phá dữ liệu trên dữ liệu lớn. (Data mining, Machine learning, AI,...)
    \item Tìm hiểu những hệ thống và hệ sinh thái dữ liệu lớn.
\end{itemize}
%-------------------------------------------------------------------------------
\chapter{ Buổi 2 }
%-------------------------------------------------------------------------------

%-------------------------------------------------------------------------------
\end{document}
