\documentclass[a4paper, 13pt]{extreport}
\usepackage{amsmath}
\usepackage{amssymb}
\usepackage{amsthm}
\usepackage{graphicx}
\usepackage{fdsymbol}
\usepackage{enumerate}
\usepackage[left=1cm,right=1cm,top=1cm,bottom=1cm]{geometry}
\usepackage{hyperref}
\usepackage[utf8]{vietnam}  
\usepackage{type1cm}                    % break the font size rule
\usepackage{times}                      % times new roman
\usepackage{fancyhdr}                   % page layout
\usepackage{pdfpages}                   % change pdf dimension
\usepackage{titlesec}                   % modified section, etc
\usepackage{tikz}                       % draw title page border
\usetikzlibrary{calc}                   % draw title page border
\usepackage{array}
\usepackage{tabularx}
\usepackage{booktabs}
\usepackage{ltablex}
\usepackage{diagbox}
\usepackage{slashbox}
\usepackage{xurl}                       % using for large URL

\setlength{\parindent}{1.5cm}
\addtocontents{toc}{\protect\thispagestyle{empty}}
\pdfpagewidth=\paperwidth
\pdfpageheight=\paperheight
\renewcommand{\labelitemii}{\(\smallcircle\)}
\newcommand{\TitleName}{Introduction to Steganography and its Applications}
\newcommand{\student}{Index Huynh}
\newcommand{\Subject}{IE406.E31}
\newcommand{\Teacher}{Lecturer: M.S. Nghi Hoàng Khoa}
\newcommand{\FontB}[3]
    {\normalfont\bfseries\fontsize{#1}{#2}\selectfont #3}
\newcommand{\FontN}[3]
    {\normalfont\fontsize{#1}{#2}\selectfont #3}
\newcommand{\maintitle}
    {\FontB{18pt}{20pt}{\Subject} \\ \FontB{18pt}{20pt}{\TitleName} \\ \FontB{14pt}{20pt}{\student} \\ \FontB{14pt}{20pt}{\Teacher}}

\newcommand{\bottitle}
    {\FontB{13pt}{20pt}{HCMC, \today}}

\newcommand{\bordertitle}{
    \begin{tikzpicture}[remember picture,overlay,inner sep=0,outer sep=0]
        \draw[black,line width=2.5pt] 
            ([xshift=-0.9cm,yshift=-0.9cm]current page.north east) coordinate (A)--
            ([xshift= 0.9cm,yshift=-0.9cm]current page.north west) coordinate (B)--
            ([xshift= 0.9cm,yshift= 0.9cm]current page.south west) coordinate (C)--
            ([xshift=-0.9cm,yshift= 0.9cm]current page.south east) coordinate (D)--cycle;
            
        \draw[black,line width=0.5pt] 
            ([xshift=-1cm,yshift=-1cm]current page.north east) coordinate (A)--
            ([xshift= 1cm,yshift=-1cm]current page.north west) coordinate (B)--
            ([xshift= 1cm,yshift= 1cm]current page.south west) coordinate (C)--
            ([xshift=-1cm,yshift= 1cm]current page.south east) coordinate (D)--cycle;
    \end{tikzpicture}
}

\newcommand{\tab}{\hspace{\parindent}}
%-------------------------------------------------------------------------------
\begin{document}
\begin{titlepage}
\bordertitle
\begin{center}
\vspace{\fill} \maintitle \\ \vspace{\fill} \bottitle
\end{center}
\end{titlepage}
%-------------------------------------------------------------------------------
\newpage
\newgeometry{left=3cm,right=2cm,top=1cm,bottom=2cm}
\tableofcontents
\thispagestyle{empty}
%-------------------------------------------------------------------------------
\newpage
\chapter*{ LỜI MỞ ĐẦU }
\thispagestyle{empty}
%-------------------------------------------------------------------------------
\vspace{\fill}
\begin{itemize}
	\item Tuân thủ pháp luật trước khi học.
	\item Học cách tư duy và sơ đồ phát triển.
	\item Cần thực hành nhiều.
	\item Các công cụ nguyên thủy được viết trên Scala (ngôn ngữ hướng hàm).
	\item Mỗi lớp thực hành 2 buổi, sau đó báo cáo cuối kỳ (trong 4 buổi).
\end{itemize}
\vspace{\fill}
%-------------------------------------------------------------------------------
\newpage
\setcounter{page}{1}
\chapter{ Giới thiệu }
%-------------------------------------------------------------------------------
\section{ Mô tả môn học }
\begin{itemize}
    \item \textbf{Kiến thức cần thiết}: Toán rời rạc và toán thống kê.
    \item \textbf{Kiến thức được học}: Nền tảng dữ liệu lớn, mô hình và công 
    cụ xử lý.
    \item \textbf{Kiến thức tự học:} Văn hóa doanh nghiệp.
    \item \textbf{Ngôn ngữ thực hành:} Python.
\end{itemize}
\section{ Mục tiêu }
\begin{itemize}
    \item \textbf{Hiểu và trình bày được các khái niệm cơ bản.}
    \item Tổ chức và lưu trữ được các nền tảng lưu trữ dữ liệu lớn.
    \item Áp dụng giải thuật Data mining, Machine learning, AI,... vào công việc.
    \item Tìm hiểu hệ thống và hệ sinh thái công ty lựa chọn.
\end{itemize}
\section{ Nội dung môn học }
\begin{itemize}
	\item Tổng quan: Lý do học dữ liệu lớn.
	\item Mô hình Mapreduce: Là mô hình ban đầu, tiền đề để phát triển.
	\item Framework Apache Spark: học kiến thức nền tảng về công cụ xử lý dữ 
	liệu lớn.
	\item Hệ sinh thái dữ liệu lớn: vừa là đồ án cuối kỳ vừa là seminar.
\end{itemize}
\section{ Phân bố thời lượng }
\begin{itemize}
	\item Buổi 1: Giới thiệu
	\item Buổi 2: Giới thiệu (tiếp)
	\item Buổi 3: MapReduce
	\item Buổi 4: MapReduce (tiếp)
	\item Buổi 5: Apache Spark
	\item Buổi 6: Apache Spark (tiếp)
	\item Buổi 7: Thực hành 1 (lớp 1): Cài đặt Apache Spark và dùng thử
	\item Buổi 8: Thực hành 1 (lớp 2): Cài đặt Apache Spark và dùng thử
	\item Buổi 9: Thực hành 2 (lớp 1): Lập trình nâng cao với Apache Spark
	\item Buổi 10: Thực hành 2 (lớp 2): Lập trình nâng cao với Apache Spark
	\item Buổi 11: Thuyết trình: Hệ sinh thái dữ liệu lớn
	\item Buổi 12: Thuyết trình: Hệ sinh thái dữ liệu lớn (tiếp)
	\item Buổi 13: Thuyết trình: Hệ sinh thái dữ liệu lớn (tiếp)
	\item Buổi 14: Thuyết trình: (tiếp) + ôn tập
\end{itemize}
\section{ Đánh giá }
\tab Thực hành 20\%, Semina + Bài tập 30\%, Thi cuối ký 50\% 
\section{ Seminar}
\begin{itemize}
	\item Tìm hiểu một hệ quản trị dữ liệu lớn
	\item Tìm hiểu hệ thống xử lý bài toán dữ liệu lớn. (Real-time)
\end{itemize}
%-------------------------------------------------------------------------------
\chapter{ Tổng quan }
\section{ Giới thiệu }
\subsection{ Bối cảnh }
\tab Hãy mua trang thiết bị từ Samsung Smart Home để sống dễ hơn (tự nấu ăn tốn 
3 tiếng với món khó thì nay chỉ tốn 1 tiếng, còn lại tự động hết), chưa kể đến
rửa chén và dọn dẹp nhà cửa. Việc gì khó, có Samsung AI lo! Và dữ liệu lớn
đã phát triển thành rất nhiều nhánh nhỏ như:
\begin{itemize}
	\item Tính toán song song
	\item Lưu trữ khối lượng dữ liệu khổng lồ
	\item Phân tán dữ liệu
	\item Kết nối mạng tốc độ cao
	\item Máy tính hiệu suất cao
	\item Quản lý tác vụ và luồng xử lý
	\item Phân tích và khai thác dữ liệu
	\item Thu thập dữ liệu
	\item Học máy
	\item Trực quan hóa dữ liệu
	\item Đều là những công nghệ rất cũ
\end{itemize}

\subsection{ Quá trình phát triển }
\subsection{ Dữ liệu hiện tại }
%-------------------------------------------------------------------------------
\section{ Định nghĩa }
%-------------------------------------------------------------------------------
\section{ Đặc điểm }
%-------------------------------------------------------------------------------
\section{ Phân loại }
%-------------------------------------------------------------------------------
\section{ Quy trình xử lý }
%-------------------------------------------------------------------------------
\section{ Ứng dụng và xu hướng }
%-------------------------------------------------------------------------------
\section{ Thách thức và cơ hội }
%-------------------------------------------------------------------------------
\section{ Q \& A }
%-------------------------------------------------------------------------------

%-------------------------------------------------------------------------------
\end{document}
